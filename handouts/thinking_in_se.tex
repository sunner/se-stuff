% $Header: /Users/joseph/Library/texmf/tex/latex/beamer/solutions/generic-talks/generic-ornate-15min-45min.en.tex,v 90e850259b8b 2007/01/28 20:48:30 tantau $

\documentclass[CJK]{beamer}

% This file is a solution template for:

% - Giving a talk on some subject.
% - The talk is between 15min and 45min long.
% - Style is ornate.



% Copyright 2004 by Till Tantau <tantau@users.sourceforge.net>.
%
% In principle, this file can be redistributed and/or modified under
% the terms of the GNU Public License, version 2.
%
% However, this file is supposed to be a template to be modified
% for your own needs. For this reason, if you use this file as a
% template and not specifically distribute it as part of a another
% package/program, I grant the extra permission to freely copy and
% modify this file as you see fit and even to delete this copyright
% notice. 


\mode<presentation>
{
  \usetheme{Warsaw}
  % or ...

  \setbeamercovered{transparent}
  % or whatever (possibly just delete it)
}


\usepackage{CJKutf8}
% or whatever

\usepackage{times}
\usepackage[T1]{fontenc}
% Or whatever. Note that the encoding and the font should match. If T1
% does not look nice, try deleting the line with the fontenc.


\title[Thinking in SE]
{软件工程思想}

\subtitle
{软件工程导论之五} % (optional)

\author[\url{http://sunner.cn}] % (optional, use only with lots of authors)
{王宇颖 \and 孙志岗}
% - Use the \inst{?} command only if the authors have different
%   affiliation.

\institute[哈尔滨工业大学] % (optional, but mostly needed)
{
  计算机科学与技术学院\\
  哈尔滨工业大学}
% - Use the \inst command only if there are several affiliations.
% - Keep it simple, no one is interested in your street address.

\date[Short Occasion] % (optional)
{2011-10-10}

\subject{Slides}
% This is only inserted into the PDF information catalog. Can be left
% out. 



% If you have a file called "university-logo-filename.xxx", where xxx
% is a graphic format that can be processed by latex or pdflatex,
% resp., then you can add a logo as follows:

% \pgfdeclareimage[height=0.5cm]{university-logo}{university-logo-filename}
% \logo{\pgfuseimage{university-logo}}



% Delete this, if you do not want the table of contents to pop up at
% the beginning of each subsection:
\AtBeginSubsection[]
{
  \begin{frame}<beamer>{主要内容}
    \tableofcontents[currentsection,currentsubsection]
  \end{frame}
}


% If you wish to uncover everything in a step-wise fashion, uncomment
% the following command: 

%\beamerdefaultoverlayspecification{<+->}


\begin{document}
\begin{CJK*}{UTF8}{gkai}

\begin{frame}
  \titlepage
\end{frame}

\begin{frame}{主要内容}
  \tableofcontents
  % You might wish to add the option [pausesections]
\end{frame}


% Since this a solution template for a generic talk, very little can
% be said about how it should be structured. However, the talk length
% of between 15min and 45min and the theme suggest that you stick to
% the following rules:  

% - Exactly two or three sections (other than the summary).
% - At *most* three subsections per section.
% - Talk about 30s to 2min per frame. So there should be between about
%   15 and 30 frames, all told.

\section{好的软件}

\subsection{什么样的软件是好软件?}

\newenvironment{iconblock}[2]{
    \begin{block}{#1}
    \begin{center}
    \includegraphics[height=2.4cm]{#2}}
{\end{center}
    \end{block}
}

\begin{frame}{下面哪个软件好?为什么?}
  % - A title should summarize the slide in an understandable fashion
  %   for anyone how does not follow everything on the slide itself.
  \begin{columns}[T]
    \begin{column}{4cm}
      \begin{iconblock}{腾讯QQ}{qq.jpg}\end{iconblock}
      \begin{iconblock}{植物大战僵尸}{pvz.jpg}\end{iconblock}
    \end{column}
    \begin{column}{4cm}
      \begin{iconblock}{Internet Explorer}{ie.png}\end{iconblock}
      \begin{iconblock}{Microsoft Word}{word.png}\end{iconblock}
    \end{column}
  \end{columns}

\end{frame}

\subsection{好软件的特质}

\begin{frame}{用户如何评价软件}
  \begin{itemize}
    \item 是否能完成我想要的工作?结果正确吗?\structure{[正确性]}
    \pause
    \item<2-> 学起来容易吗?用起来方便吗?\structure{[易用性]}
    \pause
    \item<3-> 速度快不快?吃内存吗?\structure{[性能]}
    \pause
    \item<4-> 黑客能侵入它吗?后门有木有?\structure{[安全性]}
    \pause
    \item<5-> 经常死机吗?\structure{[稳定性]}
    \pause
    \item<6-> 点卡贵不贵?装备怎么卖?\structure{[成本]}
  \end{itemize}
\end{frame}

\begin{frame}{内行如何评价软件}
  \begin{block}{}
    内行也关注用户所关注的,但还有更多\dots
  \end{block}
  \begin{columns}[t]
    \begin{column}{6.1cm}
      \begin{itemize}
        \item 容易修改和扩充吗?\structure{[扩展性]}
          \begin{itemize}
            \item 用户需求总是变更
            \item 竞争对手给的压力山大
          \end{itemize}
        \pause
        \item<2-> 代码其它软件能用吗?\structure{[复用性]}
          \begin{itemize}
            \item 降低成本
            \item 代码更可靠
          \end{itemize}
        \pause
        \item<3-> 做了充分测试吗?\structure{[可测性]}
        \pause
      \end{itemize}
    \end{column}
    \begin{column}{6.1cm}
      \begin{itemize}
        \item<4-> 新人能快速融入吗?\structure{[可维护]}
          \begin{itemize}
            \item 开发人员流动快
            \item 小白经验不足
          \end{itemize}
        \pause
        \item<5-> 能在多平台使用吗?\structure{[可移植性]}
          \begin{itemize}
            \pause
            \item Windows, Linux, Mac\dots
            \pause
            \item IE, Firefox, Chrome\dots
            \pause
            \item PC, iPhone, iPad, Android, PSP\dots
          \end{itemize}
        \pause
        \item<9-> \dots
      \end{itemize}
    \end{column}
  \end{columns}
\end{frame}

\begin{frame}{好软件的特质}
  \begin{block}{}
    \begin{center}
      \LARGE 用户和内行都说好,才是真的好
    \end{center}
  \end{block}
  \pause
  \begin{block}{}
    \begin{center}
      用户和内行矛盾了怎么办?
    \end{center}
  \end{block}
\end{frame}

\begin{frame}{应对矛盾}
  \begin{alertblock}{}
    \begin{center}
      \Huge 折中!
    \end{center}
  \end{alertblock}
\end{frame}

\section{软工核心思想}

\begin{frame}{Summary}

  % Keep the summary *very short*.
  \begin{itemize}
  \item
    The \alert{first main message} of your talk in one or two lines.
  \item
    The \alert{second main message} of your talk in one or two lines.
  \item
    Perhaps a \alert{third message}, but not more than that.
  \end{itemize}
  
  % The following outlook is optional.
  \vskip0pt plus.5fill
  \begin{itemize}
  \item
    Outlook
    \begin{itemize}
    \item
      Something you haven't solved.
    \item
      Something else you haven't solved.
    \end{itemize}
  \end{itemize}
\end{frame}


\end{CJK*}
\end{document}


